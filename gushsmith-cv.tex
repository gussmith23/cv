\documentclass[letterpaper]{article}

\usepackage[T1]{fontenc}
\usepackage{lmodern}
\usepackage{hyperref}
\usepackage{enumitem}
\usepackage{changepage} % for adjustwidth
\usepackage{bibentry} % for publications

\usepackage{geometry}
\geometry{
	letterpaper,
	left=15mm,
	top=1in,
}

% Parameters
\newcommand{\primaryindent}{1.5cm} % the primary level of indentation that most everything is indented to
\setlength{\parindent}{0cm} % first-line indent for paragraphs

% Redefine section style
\usepackage{sectsty}
\sectionfont{
	\normalfont\sffamily\large
	\sectionrule{11pt}{1pt}{0pt}{0pt}}


\newcommand{\resumesubsection}[1]{
	\vspace{1.5ex}
	\noindent \large \textsf{{#1}}\par
	\vspace{.5ex}
	\normalsize
}

\newcommand{\name}[1]{ 
	\Huge  \textsf{\textbf{#1}}
	\vspace{0.1em}
	\par \normalsize \normalfont}

\newcommand{\personalinfo}[1]{
	\large  #1 
	\par \normalsize \normalfont}

% this environment indents to \primaryindent.
\newenvironment{indented}{\begin{adjustwidth}{\primaryindent}{0em}}{\end{adjustwidth}}

% this environment is used to generate lists in this resume.
\newenvironment{resumelist}{\begin{itemize}[topsep=0pt,noitemsep,itemindent=-15pt,leftmargin=30pt]}{\end{itemize}}

\newcommand{\generalentry}[5]{
	\begin{indented}
		\Large \textsf{\textbf{#1}} \hfill	% Study
			\hfill\normalsize\textit{#2} \par  			% Location
		\noindent \large \textsf{\textbf{\textit{#3}}} 	% Subtitle
			\hfill \normalsize #4\par 					% Time
		\normalsize \normalfont #5 \par					% Rest of content
		\normalsize \normalfont
	\end{indented}
	}

\begin{document}
	
\pagenumbering{gobble} % no page numbering
	
\begin{center}
	\name{Gus Henry Smith}
	\personalinfo{107 NE 43rd St Apt 3, Seattle, WA 98105}
	\personalinfo{\url{http://justg.us} -- (570)817-9340 -- \href{mailto:gussmith@cs.washington.edu}{gussmith@cs.washington.edu}}
\end{center}

\section*{Education}

\generalentry{University of Washington}{Seattle, WA}{Paul G.~Allen School of Computer Science and Engineering}{Sept.~2018--present}{
	\begin{resumelist}
		\item Ph.D. in Computer Science and Engineering with a focus in Computer Architecture
		\item Advisor: Luis Ceze
	\end{resumelist}
}

\vspace{2mm}

\generalentry{The Pennsylvania State University}{University Park, PA}{The Schreyer Honors College}{Graduated May 2018}{
	% TODO handle margin in educationentry command
	% TODO handle topsep there too
	\begin{resumelist}
		\item B.S/M.S.~in Computer Science and Engineering
		\item Thesis: \textit{Designing Processing in Memory Architectures via Static Analysis of Real Programs}
	\end{resumelist}
}

\section*{Research Experience}


\generalentry{University of Washington SAMPL Lab}{Seattle, WA}{Researcher}{Sept.~2018--present}{		
  \begin{resumelist}
    \item Bring custom datatype support into TVM, an open-source deep learning compiler stack
  \end{resumelist}
}

\vspace{2mm}

\generalentry{Microsystems Design Lab at Penn State}{University Park, PA}{Researcher}{May 2014--May 2018}{		
	
	\resumesubsection{NSF/SRC E2CDA Center: \textit{Extremely Energy Efficient Collective Electronics}}
	\begin{resumelist}
		\item Accelerated computer vision algorithms using analog computational abilities of emerging devices
		\item\textbf{Co-authored publication:} A FerroFET Based In-Memory Processor for Solving Distributed and Iterative Optimization via Least-Squares Method \cite{ferrofet}
		\item\textbf{Co-authored publication:} Computing With Networks of Oscillatory Dynamical Systems \cite{8565896}
	\end{resumelist}
	
	\resumesubsection{NSF Expedition in Computing: \textit{Visual Cortex on Silicon}}
	
	Organizational co-lead of the 15-person Grocery Automation System group
	\begin{resumelist}
		\item Orchestrated the creation of a system to assist the visually impaired in grocery shopping, combining a number of lab projects
%		\item Organized weekly meetings and larger demos
		\item\textbf{Co-authored publication:} Third Eye: A Shopping Assistant for the Visually Impaired \cite{7842859}
	\end{resumelist}
	
	\vspace{1mm}
	
	Technical team lead of the 5-person Haptic Glove project (2014--2017)
	\begin{resumelist}
		\item Designed a haptic feedback glove which guides a user's hand towards an object utilizing computer vision library OpenCV to track objects and extrapolate the hand's position
%		\item Planned, printed, and soldered custom PCBs for the glove	
%		\item Presented a central demo annually at NSF Expedition review
%		\item Recruited, trained, and organized a team of undergraduates
	\end{resumelist}


}

%\vspace{3mm}

\section*{Engineering Experience}
\generalentry{Google}{San Francisco, CA}{Software Engineering Intern on Fuchsia}{May--August 2018}{
	\begin{resumelist}
		\item Implemented the RFCOMM Bluetooth protocol for Fuchsia, Google's new operating system
	\end{resumelist}
}
\vspace{3mm} % TODO is there a better way to do this?

\generalentry{Google}{Seattle, WA}{Software Engineering Intern on Chrome}{May--August 2017}{
	\begin{resumelist}
		\item Gained C++ and general software engineering proficiency
	\end{resumelist}
}
\vspace{3mm}
\generalentry{Automated Ceiling Tile Measurement Tool}{University Park, PA}{Team Lead, Penn State Engineering Capstone}{January--May 2017}{
	\begin{resumelist}
		\item Led a team in designing a quick and accurate app for measuring and cutting ceiling tiles
		\item Won third place overall best project at semesterly Capstone Design Showcase Awards 
	\end{resumelist}
}

\vspace{3mm}


\generalentry{Google}{Mountain View, CA}{Software Engineering Intern, Android Internal Tools}{May--August 2016}{
	\begin{resumelist}
		\item Committed code into the closed- and open-source Android trees, reaching a wide audience of both internal and external developers
	\end{resumelist}
}

\section*{Teaching Experience}

\generalentry{CMPEN 417: FPGA Design}{Penn State CSE Department}{Teaching Assistant}{Spring 2017}{
	\begin{resumelist}
		\item Created from scratch a number of course projects for the Zedboard, utilizing the board's Xilinx 7-series FPGA and ARM processor; lectured on each project
		\item Mentored and taught students one-on-one in weekly office hours
		\item Assisted in the creation and grading of all course exams and projects
	\end{resumelist}
}

\vspace{3mm}

\generalentry{CMPEN 475: Functional Verification}{Penn State CSE Department}{Teaching Assistant}{Fall 2016}{
	\begin{resumelist}
		\item As above: created new projects and lectures, held weekly office hours, graded exams and projects
	\end{resumelist}
}

\section*{Honors and Accolades}
\begin{indented}
	\begin{resumelist}
  \item Allen School Computer Science \& Engineering Research Fellowship for 2018--2019 academic year
	\item CSE Department Grad Teaching Assistant Award for Fall 2016/Spring 2017
%	\item Second place, IEEE Computer Society Global Student Challenge 2017, for submission ``Computer Vision for Good''	
%	\item Third place overall best project, Penn State College of Engineering Capstone Design Showcase Awards 2017, for project ``Automated Ceiling Tile Measurement Tool''
%	\item Evan Pugh Junior Award: given to juniors in top 0.5\% of class
	\end{resumelist}
\end{indented}

\nocite{*}
\bibliographystyle{plain}
\renewcommand{\refname}{Publications}
\bibliography{bibliography}

\end{document}
